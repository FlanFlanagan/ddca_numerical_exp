\documentclass[11pt,letterpaper]{article}
\usepackage[utf8]{inputenc}
\usepackage{caption} % for table captions
\usepackage{amsmath} % for multi-line equations and piecewises
\DeclareMathOperator{\sign}{sign}
\usepackage{graphicx}
\usepackage{relsize}
%\usepackage{textcomp}
\usepackage{xspace}
\usepackage{verbatim} % for block comments
%\usepackage{subfig} % for subfigures
\usepackage{enumitem} % for a) b) c) lists
\newcommand{\Cyclus}{\textsc{Cyclus}\xspace}%
\newcommand{\Cycamore}{\textsc{Cycamore}\xspace}%
\newcommand{\deploy}{\texttt{d3ploy}\xspace}%
\usepackage{tabularx}
\usepackage{color}
\usepackage[acronym,toc]{glossaries}
%\newacronym{<++>}{<++>}{<++>}
%\newacronym{<++>}{<++>}{<++>}
\newacronym[longplural={metric tons of heavy metal}]{MTHM}{MTHM}{metric ton of heavy metal}
\newacronym{ABM}{ABM}{agent-based modeling}
\newacronym{AHTR}{AHTR}{Advanced High Temperature Reactor}
\newacronym{ANDRA}{ANDRA}{Agence Nationale pour la gestion des D\'echets RAdioactifs, the French National Agency for Radioactive Waste Management}
\newacronym{ANL}{ANL}{Argonne National Laboratory}
\newacronym{API}{API}{application programming interface}
\newacronym{ARCH}{ARCH}{autoregressive conditional heteroskedastic}
\newacronym{ARE}{ARE}{Aircraft Reactor Experiment}
\newacronym{ARFC}{ARFC}{Advanced Reactors and Fuel Cycles}
\newacronym{ARMA}{ARMA}{autoregressive moving average}
\newacronym{ASME}{ASME}{American Society of Mechanical Engineers}
\newacronym{ATWS}{ATWS}{Anticipated Transient Without Scram}
\newacronym{BDBE}{BDBE}{Beyond Design Basis Event}
\newacronym{BIDS}{BIDS}{Berkeley Institute for Data Science}
\newacronym{CAFCA}{CAFCA}{ Code for Advanced Fuel Cycles Assessment }
\newacronym{CEA}{CEA}{Commissariat \`a l'\'Energie Atomique et aux \'Energies Alternatives}
\newacronym{CI}{CI}{continuous integration}
\newacronym{CNERG}{CNERG}{Computational Nuclear Engineering Research Group}
\newacronym{COSI}{COSI}{Commelini-Sicard}
\newacronym{COTS}{COTS}{commercial, off-the-shelf}
\newacronym{CSNF}{CSNF}{commercial spent nuclear fuel}
\newacronym{CTAH}{CTAHs}{Coiled Tube Air Heaters}
\newacronym{CUBIT}{CUBIT}{CUBIT Geometry and Mesh Generation Toolkit}
\newacronym{DAG}{DAG}{directed acyclic graph}
\newacronym{DANESS}{DANESS}{Dynamic Analysis of Nuclear Energy System Strategies}
\newacronym{DBE}{DBE}{Design Basis Event}
\newacronym{DESAE}{DESAE}{Dynamic Analysis of Nuclear Energy Systems Strategies}
\newacronym{DHS}{DHS}{Department of Homeland Security}
\newacronym{DOE}{DOE}{Department of Energy}
\newacronym{DRACS}{DRACS}{Direct Reactor Auxiliary Cooling System}
\newacronym{DRE}{DRE}{dynamic resource exchange}
\newacronym{DSNF}{DSNF}{DOE spent nuclear fuel}
\newacronym{DYMOND}{DYMOND}{Dynamic Model of Nuclear Development }
\newacronym{EBS}{EBS}{Engineered Barrier System}
\newacronym{EDZ}{EDZ}{Excavation Disturbed Zone}
\newacronym{EPA}{EPA}{Environmental Protection Agency}
\newacronym{EP}{EP}{Engineering Physics}
\newacronym{FCO}{FCO}{Fuel Cycle Options}
\newacronym{FCT}{FCT}{Fuel Cycle Technology}
\newacronym{FEHM}{FEHM}{Finite Element Heat and Mass Transfer}
\newacronym{FEPs}{FEPs}{Features, Events, and Processes}
\newacronym{FHR}{FHR}{Fluoride-Salt-Cooled High-Temperature Reactor}
\newacronym{FLiBe}{FLiBe}{Fluoride-Lithium-Beryllium}
\newacronym{GCAM}{GCAM}{Global Change Assessment Model}
\newacronym{GDSE}{GDSE}{Generic Disposal System Environment}
\newacronym{GDSM}{GDSM}{Generic Disposal System Model}
\newacronym{GENIUSv1}{GENIUSv1}{Global Evaluation of Nuclear Infrastructure Utilization Scenarios, Version 1}
\newacronym{GENIUSv2}{GENIUSv2}{Global Evaluation of Nuclear Infrastructure Utilization Scenarios, Version 2}
\newacronym{GENIUS}{GENIUS}{Global Evaluation of Nuclear Infrastructure Utilization Scenarios}
\newacronym{GPAM}{GPAM}{Generic Performance Assessment Model}
\newacronym{GRSAC}{GRSAC}{Graphite Reactor Severe Accident Code}
\newacronym{GUI}{GUI}{graphical user interface}
\newacronym{HLW}{HLW}{high level waste}
\newacronym{HPC}{HPC}{high-performance computing}
\newacronym{HTC}{HTC}{high-throughput computing}
\newacronym{HTGR}{HTGR}{High Temperature Gas-Cooled Reactor}
\newacronym{IAEA}{IAEA}{International Atomic Energy Agency}
\newacronym{INL}{INL}{Idaho National Laboratory}
\newacronym{JFNK}{JFNK}{Jacobian-Free Newton Krylov}
\newacronym{LANL}{LANL}{Los Alamos National Laboratory}
\newacronym{LBNL}{LBNL}{Lawrence Berkeley National Laboratory}
\newacronym{LCOE}{LCOE}{levelized cost of electricity}
\newacronym{LDRD}{LDRD}{laboratory directed research and development}
\newacronym{LFR}{LFR}{Lead-Cooled Fast Reactor}
\newacronym{LLNL}{LLNL}{Lawrence Livermore National Laboratory}
\newacronym{LMFBR}{LMFBR}{Liquid-Metal-cooled Fast Breeder Reactor}
\newacronym{LOFC}{LOFC}{Loss of Forced Cooling}
\newacronym{LOHS}{LOHS}{Loss of Heat Sink}
\newacronym{LOLA}{LOLA}{Loss of Large Area}
\newacronym{LP}{LP}{linear program}
\newacronym{MARKAL}{MARKAL}{MARKet and ALlocation}
\newacronym{MA}{MA}{minor actinide}
\newacronym{MCNP}{MCNP}{Monte Carlo N-Particle code}
\newacronym{MILP}{MILP}{mixed-integer linear program}
\newacronym{MIT}{MIT}{the Massachusetts Institute of Technology}
\newacronym{MOAB}{MOAB}{Mesh-Oriented datABase}
\newacronym{MOOSE}{MOOSE}{Multiphysics Object-Oriented Simulation Environment}
\newacronym{MOX}{MOX}{mixed oxide}
\newacronym{MSBR}{MSBR}{Molten Salt Breeder Reactor}
\newacronym{MSRE}{MSRE}{Molten Salt Reactor Experiment}
\newacronym{MSR}{MSR}{Molten Salt Reactor}
\newacronym{NAGRA}{NAGRA}{National Cooperative for the Disposal of Radioactive Waste}
\newacronym{NEAMS}{NEAMS}{Nuclear Engineering Advanced Modeling and Simulation}
\newacronym{NEUP}{NEUP}{Nuclear Energy University Programs}
\newacronym{NFCSim}{NFCSim}{Nuclear Fuel Cycle Simulator}
\newacronym{NFC}{NFC}{Nuclear Fuel Cycle}
\newacronym{NGNP}{NGNP}{Next Generation Nuclear Plant}
\newacronym{NNSA}{NNSA}{National Nuclear Security Administration}
\newacronym{NQA1}{NQA-1}{Nuclear Quality Assurance - 1}
\newacronym{NRC}{NRC}{Nuclear Regulatory Commission}
\newacronym{NSF}{NSF}{National Science Foundation}
\newacronym{NSSC}{NSSC}{Nuclear Science and Security Consortium}
\newacronym{NUWASTE}{NUWASTE}{Nuclear Waste Assessment System for Technical Evaluation}
\newacronym{NWTRB}{NWTRB}{Nuclear Waste Technical Review Board}
\newacronym{OCRWM}{OCRWM}{Office of Civilian Radioactive Waste Management}
\newacronym{ORION}{ORION}{ORION}
\newacronym{ORNL}{ORNL}{Oak Ridge National Laboratory}
\newacronym{PARCS}{PARCS}{Purdue Advanced Reactor Core Simulator}
\newacronym{PBAHTR}{PB-AHTR}{Pebble Bed Advanced High Temperature Reactor}
\newacronym{PBFHR}{PB-FHR}{Pebble-Bed Fluoride-Salt-Cooled High-Temperature Reactor}
\newacronym{PEI}{PEI}{Peak Environmental Impact}
\newacronym{PH}{PRONGHORN}{PRONGHORN}
\newacronym{PRKE}{PRKE}{Point Reactor Kinetics Equations}
\newacronym{PSPG}{PSPG}{Pressure-Stabilizing/Petrov-Galerkin}
\newacronym{PWAR}{PWAR}{Pratt and Whitney Aircraft Reactor}
\newacronym{PyNE}{PyNE}{Python toolkit for Nuclear Engineering}
\newacronym{PyRK}{PyRK}{Python for Reactor Kinetics}
\newacronym{QA}{QA}{quality assurance}
\newacronym{RDD}{RD\&D}{Research Development and Demonstration}
\newacronym{RD}{R\&D}{Research and Development}
\newacronym{RELAP}{RELAP}{Reactor Excursion and Leak Analysis Program}
\newacronym{RIA}{RIA}{Reactivity Insertion Accident}
\newacronym{RIF}{RIF}{Region-Institution-Facility}
\newacronym{SFR}{SFR}{Sodium-Cooled Fast Reactor}
\newacronym{SINDAG}{SINDA{\textbackslash}G}{Systems Improved Numerical Differencing Analyzer $\backslash$ Gaski}
\newacronym{SKB}{SKB}{Svensk K\"{a}rnbr\"{a}nslehantering AB}
\newacronym{SNF}{SNF}{spent nuclear fuel}
\newacronym{SNL}{SNL}{Sandia National Laboratory}
\newacronym{STC}{STC}{specific temperature change}
\newacronym{SUPG}{SUPG}{Streamline-Upwind/Petrov-Galerkin}
\newacronym{SWF}{SWF}{Separations and Waste Forms}
\newacronym{SWU}{SWU}{Separative Work Unit}
\newacronym{TRISO}{TRISO}{Tristructural Isotropic}
\newacronym{TSM}{TSM}{Total System Model}
\newacronym{TSPA}{TSPA}{Total System Performance Assessment for the Yucca Mountain License Application}
\newacronym{UFD}{UFD}{Used Fuel Disposition}
\newacronym{UML}{UML}{Unified Modeling Language}
\newacronym{UOX}{UOX}{uranium oxide}
\newacronym{UQ}{UQ}{uncertainty quantification}
\newacronym{US}{US}{United States}
\newacronym{UW}{UW}{University of Wisconsin}
\newacronym{VISION}{VISION}{the Verifiable Fuel Cycle Simulation Model}
\newacronym{VV}{V\&V}{verification and validation}
\newacronym{WIPP}{WIPP}{Waste Isolation Pilot Plant}
\newacronym{YMR}{YMR}{Yucca Mountain Repository Site}

\definecolor{bg}{rgb}{0.95,0.95,0.95}
\newcolumntype{b}{X}
\newcolumntype{f}{>{\hsize=.15\hsize}X}
\newcolumntype{s}{>{\hsize=.5\hsize}X}
\newcolumntype{m}{>{\hsize=.75\hsize}X}
\newcolumntype{r}{>{\hsize=1.1\hsize}X}
\usepackage{titling}
\usepackage[hang,flushmargin]{footmisc}
\renewcommand*\footnoterule{}
\usepackage{tikz}


\usetikzlibrary{shapes.geometric,arrows}
\tikzstyle{process} = [rectangle, rounded corners, 
minimum width=1cm, minimum height=1cm,text centered, draw=black, 
fill=blue!30]
\tikzstyle{arrow} = [thick,->,>=stealth]


\graphicspath{{figures/}}
\title{DDCA: Final Report}
\author{Gwendolyn J. Chee, Roberto E. Fairhurst, 
\\ \vspace{0.5em} Robert R. Flanagan, Kathryn D. Huff}


\begin{document}
	\maketitle
	\hrule

%----------------------------------------------------------------%
\section{Introduction}
\gls{NFC} simulation scenarios are constrained objective functions. 
The objectives are systemic demands such as "1\% power growth", 
while the constraints are availability of new nuclear technology.
To effectively simulate a nuclear fuel cycle, \gls{NFC} simulators 
must bring demand responsive deployment decisions into the dynamics
of the simulation logic \cite{huff_current_2017}. 

** Why? 

Thus, a \gls{NFC} simulator should have the capability to deploy 
supporting fuel cycle facilities to meet a user-defined commodity
demand. 
While automated power production deployment is common in most fuel
cycle simulators, automated deployment of supportive fuel cycle 
facilities is non-existent. 
Instead, the user must detail the deployment timeline of all 
supporting facilities or have infinite capacity support facilities
to ensure that there is no gap in the nuclear fuel cycle supply 
chain. 
These user-defined assumptions are not an accurate reflection 
of the real world. 
This shortcoming exists also in the fuel cycle simulator, \Cyclus. 
Therefore, there is a need to develop demand-driven deployment 
capability in \Cyclus to deploy facilities to meet front-end and 
back-end demands of the fuel cycle.

The Demand-Driven Cycamore Archetype project (NEUP-FY16-10512) 
aims to develop \Cyclus's demand-driven deployment capabilities. 
The developed algorithm will be in the form of a \Cyclus 
\texttt{Institution} agent, and will deploy \texttt{Facilities} 
to meet the front-end and back-end demands of the fuel cycle.
This demand-driven deployment capability is referred to as 
\deploy. 
Its goal is to meet demand for any commodity while minimizing 
oversupply. 

** Description of using \deploy to run transition scenarios more effectively ** 

%----------------------------------------------------------------%
\section{Background}
Hello I am \Cyclus. What am I? 

%----------------------------------------------------------------%
\section{Method}
To meet the project objective, an algorithm that deploys 
facilities to meet demand for any commodity while 
minimizing oversupply was developed. 
The \texttt{Institution} agent accepts input variables that instructs
the algorithm. 

\subsection{Input Variables}
Explain the notion of demand and supply driven deployment 
and all the specific input variables. 

\subsection{Algorithm Flow}
Describe how the algorithm flows in terms of deploy solver function, 
check constraint etc. 

\subsection{Prediction Methods}
At each time step, the demand and supply for each commodity is 
predicted for the following time step. 
Based on the prediction, \deploy deploys facilities to meet the 
predicted demand. 
The demand and supply predictions are governed by four types
of algorithms: non-optimizing, time series forecasting, 
deterministic optimizing and machine learning. 
The choice of which prediction algorithm to use is a user input. 

\subsubsection{Non-Optimizing}
List and describe each algorithm 

\subsubsection{Time Series Forecasting}
List and describe each algorithm 

\subsubsection{Deterministic Optimizing}
List and describe each algorithm 

\subsubsection{Machine Learning}
List and describe each algorithm 

%----------------------------------------------------------------%
\section{Demonstration of \deploy capabilities}
The \deploy capabilities will be demonstrated through numerical
experiments. 
The numerical experiments are in the form of fuel cycle scenarios 
where the demand driving commodity, its demand curve and the 
combination of facilities in the scenario are varied. 
Each numerical experiment will be run for each prediction
algorithm. 
The prediction algorithms will be compared to determine each 
of their strengths and weaknesses. 
And how their overall performance demonstrates \deploy's 
capabilities. 

The basis of comparison for the numerical experiments are: 
the number of time steps where demand exceeds supply, residuals
and $\chi^2$ goodness of fit test. 

The numerical experiments are broken down into four types: 
front-end facility deployment, back-end facility deployment, 
closed fuel cycle and transition scenario. 

\subsection{Front-end Deployment}

\begin{table}[h]
	\centering
	\caption {Front-end Deployment Numerical Experiments}
	\label{tab:fenum}
	\begin{tabular}{|l|p{2.75cm}|p{2.5cm}|p{2.1cm}|l|}
		\hline
		\textbf{\shortstack{Test \\ Scenario}} & \textbf{\shortstack{Facilities \\ Present}} & \textbf{\shortstack{Reactor \\ Parameters}} & \textbf{\shortstack{Driving \\ Commodity}} & \textbf{\shortstack{Demand \\ Equation}}\\
		\hline
		1 & \texttt{Source}, \texttt{Sink} & - & Fuel & 1000t\\
		\hline
		2 & \texttt{Source}, \texttt{Reactor} & Cycle time: 1, Refuel time: 0 & Power & 1000t\\
		\hline
		3 & \texttt{Source}, \texttt{Reactor} & Cycle time: 3, Refuel time: 1 & Power & 1000t\\
		\hline
		4 & \texttt{Source}, \texttt{Reactor}, \texttt{Sink} & Cycle time: 1, Refuel time: 0 & Power & $10(2.5)^{t/12}$\\
		\hline
	\end{tabular}
\end{table}

\subsection{Back-end Deployment}

\begin{table}[h]
	\centering
	\caption {Back-end Deployment Numerical Experiments}
	\label{tab:benum}
	\begin{tabular}{|l|p{2.75cm}|p{2.5cm}|p{2.1cm}|l|}
		\hline
		\textbf{\shortstack{Test \\ Scenario}} & \textbf{\shortstack{Facilities \\ Present}} & \textbf{\shortstack{Reactor \\ Parameters}} & \textbf{\shortstack{Driving \\ Commodity}} & \textbf{\shortstack{Demand \\ Equation}}\\
		\hline
		5 & \texttt{Source}, \texttt{Reactor}, \texttt{Sink} & Cycle time: 1, Refuel time: 0 & Power & 1000t\\
		\hline
		6 & \texttt{Source}, \texttt{Reactor}, \texttt{Storage}, \texttt{Sink} & Cycle time: 1, Refuel time: 0 & Power & 1000t\\
		\hline
	\end{tabular}
\end{table}

\subsection{Closed Fuel Cycle}

\begin{table}[h]
	\centering
	\caption {Closed Fuel Cycle Deployment Numerical Experiment}
	\label{tab:cfcnum}
	\begin{tabular}{|l|p{2.75cm}|p{2.5cm}|p{2.1cm}|l|}
		\hline
		\textbf{\shortstack{Test \\ Scenario}} & \textbf{\shortstack{Facilities \\ Present}} & \textbf{\shortstack{Reactor \\ Parameters}} & \textbf{\shortstack{Driving \\ Commodity}} & \textbf{\shortstack{Demand \\ Equation}}\\
		\hline
		7 & \texttt{Source}, \texttt{Reactors}, \texttt{Separation Facilities}, \texttt{Mixer Facilities} & Cycle time: 1, Refuel time: 0 & Power & 1000t\\
		\hline
	\end{tabular}
\end{table}

\subsection{Transition Scenario}

\section{Transition Scenarios}


\pagebreak 
\bibliography{bibliography}

\end{document}


